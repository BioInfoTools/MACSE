\newpage
\part*{\large \centering ВВЕДЕНИЕ}
\addcontentsline{toc}{part}{ВВЕДЕНИЕ}
\hspace{\parindent} Современная биоинформатика --- это молодая, бурно развивающаяся наука, возникшая в 1976-1978 годах и окончательно оформившаяся в 1980 году со специальным выпуском журнала <<Nucleic Acid Research>> (NAR)~\cite{MironovLect}. По сути, это собрание различных математических моделей и методов в помощь биологам для решения биологических задач, таких как: предсказание пространственной структуры белков, расшифровка структуры ДНК, хранение, поиск и аннотация биологической информации.\\
\indent Основу биоинформатики составляют сравнения. Одна из ключевых задач --- поиск сходства последовательностей. Ее решение позволяет понять функциональное назначение частей геномов, оценить эволюционное расстояние между ними. Кроме этого, различия в генотипах могут объяснить различия в фенотипах.\\
\indent Для того чтобы определить, насколько две последовательности <<похожи>>, используют алгоритмы выравнивания. Они основаны на размещении исходных последовательностей мономеров ДНК, РНК или белков друг под другом таким образом, чтобы было легко увидеть их сходные участки~\cite{WikiPairAlign}. Качество выравнивания оценивают, назначая штрафы за несовпадение букв и за наличие пробелов (когда приходится раздвигать одну последовательность для того, чтобы получить наибольшее число совпадающих позиций), например, через расстояние Левенштейна --- минимальное число элементарных операций (вставка, удаление или замена символа в строке), чтобы превратить одну строку в другую~\cite{Levenshtein}. При сравнении ищется такой вариант выравнивания, чтобы итоговый счет был максимален. В такой постановке задача называется поиском <<глобального выравнивания>>. Необходимо отметить, что для полных геномов глобальное выравнивание не работает, так как при мутации, помимо вставок, удалений и замен, бывают нелинейные перестройки, которые могут менять порядок и ориентацию целых геномных блоков. Для решения, аналогично задаче поиска глобального выравнивания, формулируют задачу поиска <<локального выравнивания>>: для двух произвольных строк $A$ и $B$ найти две самые похожие подстроки и их выравнивание.\\ 
\indent Алгоритмы множественного выравнивания, аналогично алгоритмам парного выравнивания, представляют собой инструмент для установления функциональных, структурных или эволюционных взаимосвязей между биологическими последовательностями.  Несмотря на то что задача множественного выравнивания была сформулирована более 20 лет назад~\cite{SIAM_Journal}, она до сих пор не теряет своей актуальности. Если говорить о множественном глобальном выравнивании, то, по сравнению с парным выравниванием, практически ничего не меняется: необходимо расставить разрывы в выравниваемых строках таким образом, чтобы <<счет по столбцам>> был максимален. Счет по столбцу можно вести, перебирая все пары символов. Множественное локальное выравнивание обобщить на многомерный случай не так просто. Во-первых, какие-то подстроки могут быть не во всех последовательностях. Во-вторых, последовательности могут содержать дуплицированные участки. Поэтому для решения такой задачи необходимо более точно сформулировать условия выравнивания.\\ 
\indent Таким образом, две главные составляющие автоматических методов выравнивания --- это непосредственно алгоритм и функция оценки качества полученного результата. На сегодняшний день можно выделить два основных алгоритма выравнивания биологических последовательностей: алгоритм Нидлмана-Вунша и алгоритм Смита-Ватермана.  Существуют различные их модификации, использующие эвристики для уменьшения количества шагов алгоритма или требуемого объема памяти, однако,эти методы строят выравнивание без проверки биологического смысла результата. В погоне за лучшим счетом происходит потеря качества: множественные разрывы на нуклеотидном и появление стоп-кодонов на аминокислотном уровнях.\\ 
\indent Построение качественного, биологически обоснованного выравнивания нуклеотидных последовательностей с сохранением открытых рамок считывания является очень важной и пока что нерешенной задачей биоинформатики. Для получения парного выравнивания на данный момент существуют программные утилиты,выдающие хороший результат, однако, имеющие столь высокую вычислительную сложность, что не могут быть расширены для построения множественных выравниваний.\\ 
\indent Цель данного дипломного проекта --- разработка и реализация алгоритма парного выравнивания, который впоследствии можно использовать для получения множественного выравнивания. При построении ответа должна учитываться трансляция результата на уровень аминокислот. Полученное приложение сравнивается по качеству выравнивания и скорости его получения с существующим аналогом --- MACSE~\cite{MACSE}.